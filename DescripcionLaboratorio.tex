\documentclass{article}

\begin{document}

Pepito está diseñando el sistema de seguridad de su servidor. Para este
proposito, tiene una clave numerica de 4 dígitos.\\

Como Pepito es muy olvidadizo, incorporó un sistema mediante el cual su
servidor dice si la clave que ingresó es demasiado grande o demasiado chica,
asi puede intentar varias veces hasta acertar.\\

Es posible intentar conectarse al servidor de varias maquinas al mismo tiempo
pero solo le otorga acceso a una, asique aunque pepito sabe que es posible
adivinar la clave usando el sistema, no le preocupa ya que la recuerda 
vagamentey puede acertarla con pocos intentos (menos que los que necesitaria
un adversario)

Ayuda a pepito a implementar su sistema completando su programa en C usando sockets.

El sistema deberá tener una arquitectura cliente-servidor. Por un lado el cliente debe permitir 
elegir a que servidor conectars y establecer una conexión con el. Luego deberá mostrar una interfaz 
que permita hacer intentos de ingreso al servidor ingresando contraseñas, además de recibir la respuesta
del servidor, procesarla y mostrarla en pantalla antes de permitir hacer otro intento. Una vez que se 
consiga acceso al servidor debe poder enviar mensajes de texto al servidor y la posibilidad de cerrar sesion. 

El servidor debera permitir establecer una clave de acceso al iniciarse,luego deberá esperar recibir conexiones 
de distintos clientes y posibilitar que estos hagan intentos de acceso. Una vez que un cliente logre acceder debe 
rechazar intentos de acceso de otros clientes además de recibir mensajes de textos del cliente con acceso e imprimir los
en pantalla, al final el cliente cerrará sesión lo que permitirá al servidor a darle acceso a otro cliente.

Nota: El servidor debe permitir varias conexiones en simultaneo pero solo darle acceso a una sola.


\end{document}
