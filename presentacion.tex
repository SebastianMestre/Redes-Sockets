\documentclass{beamer}

\mode<presentation> {
    
\usetheme{Berlin}
\usecolortheme{beaver}

%\setbeamertemplate{footline} % To remove the footer line in all slides uncomment this line
%\setbeamertemplate{footline}[page number] % To replace the footer line in all slides with a simple slide count uncomment this line
%\setbeamertemplate{navigation symbols}{} % To remove the navigation symbols from the bottom of all slides uncomment this line
}

\usepackage{graphicx}
\usepackage{booktabs}

%----------------------------------------------------------------------------------------
%	TITLE PAGE
%----------------------------------------------------------------------------------------

\title[Sockets]{Interfaz de sockets}
\subtitle{Enviando y recibiendo datos desde un programa}
\author{S. Mestre \and M. Roma \and I. Quintero}
\institute[IPS UNR]{Instituto Politecnico Superior}
\date{\today}

\begin{document}

\begin{frame}
\titlepage
\end{frame}

\begin{frame}
\frametitle{Overview}
\tableofcontents
\end{frame}

%----------------------------------------------------------------------------------------
%	PRESENTATION SLIDES
%----------------------------------------------------------------------------------------

%------------------------------------------------
\section{Protocolos}
%------------------------------------------------

\begin{frame}
\huge{\centerline{Repaso De Protocolos}}
\end{frame}

%------------------------------------------------

\begin{frame}
    \frametitle{Repaso De Protocolos}
    \begin{columns}
        \begin{column}{0.5\textwidth}
            \textbf{TCP}
            \begin{itemize}
                \item Transmisi\'on confiable
                \item Control de flujo
                \item Orientado a la conexi\'on
                \item Bidireccional
            \end{itemize}
        \end{column}
        \begin{column}{0.5\textwidth}
            \textbf{UDP}
            \begin{center}
                \begin{itemize}
                    \item No es confiable
                    \item Sin ACKs ni retransmisiones
                    \item Sin conexi\'on
                    \item Unidireccional
                \end{itemize}
            \end{center}
        \end{column}
    \end{columns}
\end{frame}

%------------------------------------------------

\begin{frame}
%    \frametitle{}
    \begin{columns}
        \begin{column}{0.5\textwidth}
            \begin{center}
                \textbf{¿Qu\e' es un socket?}\\
                ¿Se come?
            \end{center}
        \end{column}
        \begin{column}{0.5\textwidth}
            \textbf{UDP}
            \begin{center}
                Un socket no es m\a's que una abstracci\o'n, una forma de representar un concepto en nuestro c\o'digo.\\
                Los sockets representan conexiones y ofrecen una interfaz uniforme para interactuar con ellas.
            \end{center}
        \end{column}
    \end{columns}
\end{frame}

%------------------------------------------------

\begin{frame}
%    \frametitle{}
    \begin{columns}
        \begin{column}{0.5\textwidth}
            \begin{center}
                \textbf{¿Qu\e' lo caracteriza?}\\
            \end{center}
        \end{column}
        \begin{column}{0.5\textwidth}
            \textbf{UDP}
            \begin{center}
                \begin{itemize}
                    \item Una direccion IP
                    \item Un n\u'mero de puerto
                    \item Un protocolo (como TCP o UDP)
                \end{itemize}
            \end{center}
        \end{column}
    \end{columns}
\end{frame}

%------------------------------------------------

\begin{frame}
    \frametitle{Beneficios de usar sockets}
    Utilizando sockets. podemos 'separar' nuestro programa del protocolo que usamos. De esta manera, reducimos la cantidad de codigo mientras agregamos funcionalidad\\
    Los sockets permiten intercambiar informacion entre procesos, posibilitando sistemas multi-proceso sin hacer hacks con mapeos de memoria.\\
    Mediante el uso de sockets podemos construir contratos con garantias beneficiosas.
\end{frame}

%------------------------------------------------

\begin{frame}
    \Huge{\centerline{Los sockets est\a'n b\a'rbaros}}
\end{frame}

%------------------------------------------------

\begin{frame}[fragile]
    \frametitle{Archivos ≈ Sockets ( Primera Aproximaci\'on )}
    \begin{columns}
        \begin{column}{0.5\textwidth}
            \begin{verbatim}
FILE* fp = fopen(...);



fwrite(fp, ...);
fread(fp, ...);

fclose(fp);    
            \end{verbatim}        
        \end{column}
        \begin{column}{0.5\textwidth}
            \begin{verbatim}
int sd = socket();
bind(sd, ...);
connect(sd, ...);

send(sd, ...);
recv(sd, ...);

close(sd);
            \end{verbatim}
        \end{column}
    \end{columns}
\end{frame}

%------------------------------------------------

\begin{frame}
\Huge{\centerline{Fin}}
\end{frame}

%------------------------------------------------

\end{document} 
